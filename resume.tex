%
% LaTeX source of my resume
% =========================
%
% Heavily commented to to fit even LaTeX beginners (hopefully).
%
% See the `README.md` file for more info.
%
% This file is licensed under the CC-NC-ND Creative Commons license.
%


% Start a document with the here given default font size and paper size.
\documentclass[10pt,a4paper]{article}

% Set the page margins.
\usepackage[a4paper,margin=0.69in]{geometry}

% Setup the language.
\usepackage[english]{babel}
\hyphenation{Some-long-word}

% allow utf-8
\usepackage[utf8]{inputenc}

% Makes resume-specific commands available.
\usepackage{resume}
\usepackage{footmisc}

\renewcommand{\thefootnote}{\fnsymbol{footnote}}

\begin{document}  % begin the content of the document
\sloppy  % this to relax whitespacing in favour of straight margins


% title on top of the document
\maintitle{DAVID N.P. KOPPSTEIN}{}{}

\nobreakvspace{0.1em}  % add some page break averse vertical spacing

% \noindent prevents paragraph's first lines from indenting
% \mbox is used to obfuscate the email address
% \sbull is a spaced bullet
% \href well..
% \\ breaks the line into a new paragraph
\noindent\href{mailto:david.koppstein@gmail.coml}{david.koppstein\mbox{}@\mbox{}gmail.com}\sbull
\textsmaller{}+49 170 7740947 \sbull Citizenship: Dual Australian/USA
\vspace{0.25in}

\spacedhrule{-0.2em}{-0.6em}

\roottitle{Education}

\headedsection
  {Massachusetts Institute of Technology}
  {\textsc{Cambridge, MA}} {%
  \headedsubsection
    {\textnormal{Ph.D, Biology}}
    {2015} {}
}

\headedsection
  {Yale University}
  {\textsc{New Haven, CT}} {%
  \headedsubsection
    {\textnormal{B.S., Molecular Biophysics and Biochemistry}}
    {2008} {}
}
\vspace{0.25in}

\spacedhrule{-0.2em}{-0.6em}

\roottitle{Experience}

\headedsection  % sets the header for the section and includes any subsections
  {Bioinformatics Core, MPI for Immunobiology and Epigenetics}
  {\textsc{Freiburg, Germany}} {%
  \headedsubsection
    {Staff Bioinformatician}
    {August \apo22 -- present}
    {\bodytext{Developed pipelines for demultiplexing and processing high-throughput sequencing data for end users. Spearheaded projects on transcriptome isoform annotation and chromosomal conformation capture using long reads and virus detection in single-cells.}}
}

\headedsection  % sets the header for the section and includes any subsections
  {Nikolaus Rajewsky Lab, Max Delbrück Center for Molecular Medicine}
  {\textsc{Berlin, Germany}} {%
  \headedsubsection
    {Postdoctoral Researcher}
    {June \apo18 -- June \apo22}
    {\bodytext{Conceived, wrote, and executed a Chan-Zuckerberg Initiative Neurodegeneration Grant to use 10X Visium spatial and Nanopore long read sequencing in tandem to analyze aberrant RNA isoforms in neuromuscular organoids models of Amyotrophic Lateral Sclerosis. Conceived, wrote, and executed a grant from the Berlin Institute of Health to research the effects of COVID-19 patient serum the endothelial transcriptome. Wrote and executed a grant from the Bundesministerium für Bildung und Forschung to explore the role of miR-155 and vault RNAs in SARS-CoV-2 infection. Performed sample preparation, RNA-seq library construction, and downstream bioinformatic analysis for these and other projects. Trained one Masters student. }}
}

\headedsection  % sets the header for the section and includes any subsections
  {Viral Immunology Systems Program, Kirby Institute, UNSW}
  {\textsc{Sydney, Australia}} {%
  \headedsubsection
    {Postdoctoral Researcher}
    {Apr \apo17 -- April \apo18}
    {\bodytext{Developed VDJPuzzle, a computational method for analyzing single-cell transcriptomics data of immune cells. Used this data and other methods to analyze single-cell transcriptomes of rogue B cells in Sjögren's disease. Performed single-cell RNA-seq using 10X Chromium and downstream bioinformatic analysis. Mentored two Ph.D students, a Masters student, and an Honors student. }}
}

\headedsection  % sets the header for the section and includes any subsections
  {Juno Therapeutics}
  {\textsc{Seattle, WA, USA}} {%
  \headedsubsection
    {Data Scientist}
    {Jan \apo16 -- Mar \apo17}
    {\bodytext{Principal architect of the immunosequencing bioinformatics pipeline. Scaled pipeline to the AWS cloud using CfnCluster and Snakemake. Codiscovered effective TCR binders for HPV E6 and E7, resulting in a patent. Analyzed and communicated immunosequencing analyses for internal projects and external academic collaborators. Conducted analyses for internal RNA-seq and vector integration profiling projects. Served as interim manager of the data science team during the acquisition. }}
}

\headedsection  % sets the header for a subsection and contains usually body text
  {AbVitro}
  {\textsc{Boston, MA, USA}} {%
  \headedsubsection
    {Data Scientist}
    {Mar \apo15 -- Dec \apo15}
    {\bodytext{Principal architect of the immunosequencing pipeline for single-cell immunosequencing. Mined receptor sequences from tumor-infiltrating lymphocytes and developed automated data visualization strategies to nominate candidates for screening. Communicated with external academic and industry collaborators. \textit{(Acquired by Juno Therapeutics).}}}
}

\headedsection  % sets the header for a subsection and contains usually body text
  {David Bartel Lab, MIT}
  {\textsc{Boston, MA, USA}} {%
  \headedsubsection
    {Graduate Student}
    {May \apo10 -- Mar \apo15}
    {\bodytext{Designed, implemented, and analyzed high-throughput sequencing experiments using the Illumina platform to investigate influenza's cap-snatching repertoire, discovering that small nuclear RNAs are the primary targets of this process. Utilized 3P-seq to precisely examine alternative cleavage and polyadenylation in diverse genetic backgrounds and model organisms in order to update annotations and improve microRNA targeting models. }}
}

\headedsection  % sets the header for a subsection and contains usually body text
  {Joan Steitz Lab, Yale}
  {\textsc{New Haven, CT, USA}} {%
  \headedsubsection
    {Undergraduate Student}
    {May \apo06 -- Apr \apo08}
    {\bodytext{Investigated the function of Herpesvirus saimiri Small U RNAs (HSURs) by cloning aptamers, pulldowns, and mass spectrometry.}}
}

\newpage
\roottitle{Skills}  % a root section title

  % special section that has an inline header with a 'hanging' paragraph
\headedsection{Wet lab}
{\bodytext{Extensive experience with NGS library preparation and custom design: 3P-Seq, SMART-seq, high-throughput 5' RACE, and standard RNA-seq. Experience running single-cell 10X Chromium, 10X Visium spatial sequencing, and Nanopore long-read sequencing. Experience with molecular cloning, tissue culture, Northern/Western blots, protein purification, PCR/qPCR, microscopy, cryotome sectioning, and yeast genetics.}}

\vspace{-0.3em}

\headedsection{Dry lab}
{\bodytext{Fluent in Python and R. Expertise with modern bioinformatics tools including samtools, STAR, featureCounts, edgeR, Picard, GSEA, etc. Experience with analysing single-cell \
and spatial sequencing data using scater, Seurat, napari, etc. Experience writing scalable computational pipelines using Snakemake and scaling to the cloud using AWS. Experience\
 with statistical methods and machine learning techniques. Experience managing a team of computational biologists using agile methods.} \vspace{-0.2em}
}

\vspace{0.1in}

\spacedhrule{0.1em}{-0.2em}  % a horizontal line with some vertical spacing before and after

\roottitle{Fellowships and Awards}  % a root section

\begin{itemize}
\item Juno CEO Award for Outstanding Performance, 2016
\item NSF Graduate Research Fellowship Program, Honorable Mention, 2011
\item Yale STARS II Research Fellowship, 2007-8
\end{itemize}

\spacedhrule{0.1em}{-0.2em}  % a horizontal line with some vertical spacing before and after

\roottitle{Grants}

{\noindent Wolfgang Kübler, Holger Gerhardt, Philipp Mertins, Nikolaus Rajewsky, Martin Witzenrath, \textbf{David Koppstein}, Laura Michalick, Anna Szymborska-Mell. The COVID-19 puzzle - The microvascular barrier as a mi\
ssing piece between coagulation and organ failure. Berlin Institute for Health, Vascular Biomedicine Focus Area Grant.}  \vspace{0.2em} \\

{\noindent Mina Gouti and Nikolaus Rajewsky. Spatial Analysis of aberrant RNA isoforms in ALS neuromuscular organoids. Neurodegeneration Challenge Network Collaborative Pair Pilot Project Award. (Written on behalf of Nikolaus Rajewsky)} \vspace{0.2em} \\

{\noindent Nikolaus Rajewsky, Marcel Müller, and Markus Landthaler. Molecular exploration of SARS-CoV-2 host-virus interactions with targeted knockouts and pulldowns. Bundesministerium für Bildung und Forschung, COVID-19\
 awards. (Written on behalf of Nikolaus Rajewsky)}. \vspace{0.2em} \\

\spacedhrule{0.1em}{-0.2em}  % a horizontal line with some vertical spacing before and after

\roottitle{Presentations and Posters}

{\noindent \textbf{Koppstein D}, Rizzetto S, Samir J, Singh M, Reed JH, Cai CH, Lloyd AR, Eltahla AA, Goodnow CC, Luciani F. VDJPuzzle: A computational method for BCR and TCR reconstruction from single-cell sequencing data. Australian Bioinformatics And Computational Biology Society. Oral presentation delivered in Adelaide, Australia, November 2017.} \vspace{-0.8em} \\

{\noindent \textbf{Koppstein D} Rizzetto S, Samir J, Singh M, Reed JH, Cai CH, Lloyd AR, Eltahla AA, Goodnow CC, Luciani F. VDJPuzzle: A computational method for BCR and TCR reconstruction from single-cell sequencing data. Australian Cellular Panomics Consortium. Poster presentation delivered in Melbourne, Australia, November 2017.} \vspace{-0.8em} \\

{\noindent \textbf{Koppstein D}, Rizzetto S, Samir J, Singh M, Reed JH, Cai CH, Lloyd AR, Eltahla AA, Goodnow CC, Luciani F. VDJPuzzle: A computational method for BCR and TCR reconstruction from single-cell sequencing data. Australian Society for Immunology. Oral presentation delivered in Bowral, Australia, November 2017.} \vspace{-0.8em} \\

{\noindent \textbf{Koppstein D}, Ashour J, Bartel D. Quantitative assessment of influenza's cap-snatching repertoire by RNA sequencing. RNA Society. Poster presentation delivered in Quebec City, Canada, June 2014.}

\newpage

\roottitle{Publications and Preprints}  % a root section title

{\noindent Bujanic L, Shevchuk O, von Kügelgen N, Kalinina A, Ludwik K, \textbf{Koppstein D}, Zerna N, Sickmann A, Chekulaeva M. The key features of SARS-CoV-2 leader and NSP1 required for viral escape of NSP1-mediated repression. 2022. \textit{RNA. doi: 10.1261/rna.079086.121}. \vspace{0.5em} \\ 
{\noindent Wyler M, Mösbauer K, Franke F, Diag A, Gottula LT, Arsie R, Klironomos F, \textbf{Koppstein D}, ..., Rajewsky N, Drosten C, Landthaler M. Transcriptomic profiling of SARS-CoV-2 infected human cell lines identifies HSP90 as target for COVID-19 therapy. 2021. \textit{iScience. doi: 10.1016/j.isci.2021.102151}. \vspace{0.5em} \\
{\noindent Singh M, Jackson KJL, Wang JJ, Schofield P, Field MA, \textbf{Koppstein D}, ..., Luciani F, Gordon TP, Goodnow CC, Reed JH. Lymphoma driver mutations in the pathogenic evolution of an iconic human autoantibody. 2020. \textit{Cell. doi:10.1016/j.cell.2020.01.029}. \vspace{0.5em} \\
{\noindent Rizzetto S, \textbf{Koppstein D}, Samir J, Singh M, Reed JH, Cai CH, Lloyd AR, Eltahla AA, Goodnow CC, Luciani F. B-cell receptor reconstruction from single-cell RNA-seq with VDJPuzzle. 2018. \textit{Bioinformatics. doi:10.1093/bioinformatics/bty203}. \vspace{0.5em} \\
{\noindent Grüning B, Dale R, Sjödin A, Chapman BA, Rowe J, Tomkins-Tinch CH, Valieris R, Koster J, \textbf{The Bioconda Team}. Bioconda: sustainable and comprehensive software distribution for the life sciences. 2018. \textit{Nature Methods 15, 475-76}.} \vspace{0.4em} \\
{\noindent Grigaityte K, Carter JA, Goldfless SJ, Jefferey EW, Hause RJ, Jiang Y, \textbf{Koppstein D}, Briggs AW, Church GM, Vigneault F, Atwal GS. Single-cell sequencing reveals $\alpha \beta$ chain pairing shapes the T cell repertoire. 2017. \textit{bioRxiv}.} \vspace{-0.8em} \\
{\noindent Briggs AW, Goldfless SJ, Timberlake S, Belmont BJ, Clouser CR, \textbf{Koppstein D}, Sok D, Vander Heiden JA, Tamminen MV, Kleinstein SH, Burton DR, Church GM, Vigneault F. 2017. Tumor-infiltrating immune repertoires captured by single-cell barcoding in emulsion. \textit{bioRxiv}. \vspace{0.5em} \\
{\noindent \textbf{Koppstein D}, Ashour J, and Bartel DP. Sequencing the cap-snatching repertoire of H1N1
influenza provides insight into the mechanism of viral transcription initiation. 2015. \textit{Nucleic Acids Research 43(10), 5052-5064}.} \vspace{0.5em} \\
{\noindent Hezroni H, \textbf{Koppstein D}, Schwartz M, Tabin CJ, Bartel DP, and Ulitsky I. 2015. Principles of long noncoding
RNA evolution derived from direct comparison of transcriptomes in 14 vertebrates. \textit{Cell Reports 11(7), 1110-1122}}. \vspace{0.5em} \\
{\noindent Nam J-W, Rissland OS, \textbf{Koppstein D}, Abreu-Goodger C, Jan CH, Agarwal V, Yildirim, Rodriguez A, and Bartel DP. 2014. Global Analyses of the Effect of Different Cellular Contexts on MicroRNA Targeting. \textit{Molecular Cell. 53, 1031-1043.}} \vspace{0.5em} \\
{\noindent Ulitsky I, Shkumatava A, Jan C, Subtelny AO, \textbf{Koppstein D}, Bell G, Sive H, and Bartel DP. 2012. Extensive alternative polyadenylation during zebrafish development. \textit{Genome Research. 22(10):2054-66.}} \vspace{0.5em} \\
{\noindent Agarwal A\footnote[1]{Authors contributed equally to this publication.}, \textbf{Koppstein D}\footnotemark[1], Rozowsky J, Sboner A, Habegger L, Hillier LW, Sasidharan R, Reinke V, Waterston RH, and Gerstein M. 2010. Comparison and calibration of transcriptome data from RNA-Seq and tiling arrays. \textit{BMC Genomics. 11:383.}} \vspace{0.3em} \\
{\noindent Mukhopadhyay J, Das K, Ismail S, \textbf{Koppstein D}, Jang M, Hudson B, Sarafianos S, Tuske S, Patel J, Jansen R, Irschik H, Arnold E, and Ebright RH. 2008. Myxopyronin, Corallopyronin, and Ripostatin Inhibit Transcription by Binding to the RNA Polymerase Switch Region. \textit{Cell. 135:295-307.}} \vspace{0.3em} \\

\spacedhrule{0.1em}{-0.2em}  % a horizontal line with some vertical spacing before and after

\roottitle{Extracurricular}

\begin{itemize}
\item Instructor for NGSSchool seminar, ``Introduction to Snakemake'', 2020
\item Science outreach with ``Lange Nacht für Wissenschaft'' (Long Night of Science), Berlin, 2019-2021
\item Urban Cycling Skills and Safety Clinic Organizer, 2013-4
\item Whitehead Partner for High School Science Teacher Outreach, 2012
\item MIT Cycling Team Officer, 2012-2014
\end{itemize}

\spacedhrule{0.1em}{-0.2em}  % a horizontal line with some vertical spacing before and after

\roottitle{Languages}

\begin{itemize}
\item English - native speaker
\item German - B2 level
\end{itemize}


\end{document}

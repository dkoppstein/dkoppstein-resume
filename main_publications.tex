%
% LaTeX source of my resume
% =========================
%
% Heavily commented to to fit even LaTeX beginners (hopefully).
%
% See the `README.md` file for more info.
%
% This file is licensed under the CC-NC-ND Creative Commons license.
%


% Start a document with the here given default font size and paper size.
\documentclass[10pt,a4paper]{article}

% Set the page margins.
\usepackage[a4paper,margin=0.69in]{geometry}

% Setup the language.
\usepackage[english]{babel}
\hyphenation{Some-long-word}

% allow utf-8
\usepackage[utf8]{inputenc}

% Makes resume-specific commands available.
\usepackage{resume}
\usepackage{footmisc}

\renewcommand{\thefootnote}{\fnsymbol{footnote}}

\begin{document}  % begin the content of the document
\sloppy  % this to relax whitespacing in favour of straight margins


% title on top of the document
{\centering
\maintitle{List of the selected key publications of David Koppstein}{}{}\par
}
\nobreakvspace{0.1em}  % add some page break averse vertical spacing

% \noindent prevents paragraph's first lines from indenting
% \mbox is used to obfuscate the email address
% \sbull is a spaced bullet
% \href well..
% \\ breaks the line into a new paragraph
\vspace{0.05in}

%Please explain your selection of key publications here (relevance of results,
%significance for your academic profile; maximum of 1000 characters). In cases of
%multiple authorship please also indicate your own personal contribution to this key publication.]
%
\begin{enumerate}
{\item \textbf{Koppstein D}, Ashour J, and Bartel DP. Sequencing the cap-snatching repertoire of H1N1
influenza provides insight into the mechanism of viral transcription initiation. 2015. \textit{Nucleic Acids Research 43(10), 5052-5064}.} \vspace{0.5em} \\

In this paper, I used high-throughput sequencing to comprehensively characterize influenza virus cap-snatching, a process by which the virus steals 5-methylguanosine caps from host messages. Surprisingly, I found that small nuclear RNA precursors are the most abundant substrates for influenza cap-snatching, which had not been observed previously. Further, I clarified the mechanism of viral transcription initiation, indicating that a single purine is sufficient to prime transcript initiation, and that stuttering of viral transcription is dependent on the template sequence. This paper was particularly important for my academic career since I conceived of the experiment independently of my supervisor, forged a collaboration, designed and executed the wet-lab protocol, and carried out the computational analysis (indeed, everything except for handling the virus). Thus, I proved that I can carry out an entire project from conception to completion independently.

{\item Nam J-W, Rissland OS, \textbf{Koppstein D}, Abreu-Goodger C, Jan CH, Agarwal V, Yildirim, Rodriguez A, and Bartel DP. 2014. Global Analyses of the Effect of Different Cellular Contexts on MicroRNA Targeting. \textit{Molecular Cell. 53, 1031-1043.}} \vspace{0.5em} \\

This paper demonstrated that microRNA targeting efficacy depends on the ratio of isoforms that contain effective microRNA sites, and that this ratio can vary drastically between cell types due to alternative polyadenylation. Therefore, we developed a computational method, called wContext+, to incorporate this “affected isoform ratio” into the TargetScan microRNA targeting algorithm. My role in this publication was to generate transcriptome-wide polyadenylation site usage maps using the 3P-seq protocol for the cell lines studied, which required considerable experimental optimization. Importantly, working on alternative polyadenylation in this and other papers from the Bartel lab led to my current fascination with the mechanisms of regulating alternative polyadenylation in physiologically relevant settings such as neurons, which is now my main academic pursuit. \vspace{0.5em} \\

{\item Rizzetto S, \textbf{Koppstein D}, Samir J, Singh M, Reed JH, Cai CH, Lloyd AR, Eltahla AA, Goodnow CC, Luciani F. B-cell receptor reconstruction from single-cell RNA-seq with VDJPuzzle. 2018. \textit{Bioinformatics. doi:10.1093/bioinformatics/bty203}}. \vspace{0.5em} \\

In this paper, we developed a method for reconstruction of full-length B-cell receptors chain pairs from single-cell RNA sequencing data. This is important for the immunology field, because studies of clonal evolution in single cells require B-cell receptors with single-nucleotide resolution. We extensively validated our pipeline, VDJPuzzle, using Sanger sequencing from PBMC-derived B cells. My contribution to the paper was to develop and evaluate an error correction module to improve the reconstruction accuracy, which is critically important for B cells in contrast to T cells since somatic hypermutation occurs along the full length of the chain, not just in the CDR3 region. Furthermore, I streamlined and modernized the computational pipeline and co-wrote the paper. \vspace{0.5em} \\

\end{enumerate}

\end{document}
